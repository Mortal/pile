% Mathias Rav, fredag i rusdagene d. 21. august 2015
% Udvidet 5. august 2016
%
% Men hvordan fungerer det her pilemagi?, spørger du.
%
%   \pil{\datit}
%   {Matematisk kantine}
%   {\scalebox{2}{Hej}}
% bliver til...
%   \newcommand{\datit}[2]{
%     \Repeat{#2}{
%       \skilt
%       {\includegraphics[angle=#1,origin=c]{pil.pdf}}
%       {\scalebox{5.60}[16]{Matematisk kantine}}
%       {\scalebox{2}{Hej}}}}
%
% Og så kan man skrive f.eks. \datit{\venstre}{2} som bliver til
%   \Repeat{2}{
%     \skilt
%     {\includegraphics[angle=\venstre,origin=c]{pil.pdf}}
%     {\scalebox{5.60}[16]{Matematisk kantine}}
%     {\scalebox{2}{Hej}}}


\documentclass[oneside,a4paper]{memoir}

\usepackage[T1]{fontenc}
\usepackage[utf8]{inputenc}

\usepackage[pdftex]{graphicx}
\setlength{\headheight}{0pt}
\setlength{\headsep}{0pt}
\setlength{\footskip}{0pt}
% A4: 21 cm x 29.7 cm
% 29.7 cm = 2 * 0.85 cm + 28 cm
% 28 cm = 2 * 4 cm + 20 cm
\setulmarginsandblock{0.85cm}{*}{1}
\setlrmarginsandblock{1cm}{*}{1}
\checkandfixthelayout
\pagestyle{empty}

% From http://tex.stackexchange.com/a/16194/76220
\makeatletter
\newcommand{\Repeat}[1]{%
    \expandafter\@Repeat\expandafter{\the\numexpr #1\relax}%
}

\def\@Repeat#1{%
    \ifnum#1>0
        \expandafter\@@Repeat\expandafter{\the\numexpr #1-1\expandafter\relax\expandafter}%
    \else
        \expandafter\@gobble
    \fi
}
\def\@@Repeat#1#2{%
    \@Repeat{#1}{#2}#2%
}
\makeatother
% End from tex.se

% \skilt{illustration}{navn}{tekst}
\newcommand{\skilt}[3]{
  \parbox[c][6cm][c]{\textwidth}{%
    \centering%
    #2%
  }\par
  \parbox[c][16cm][c]{\textwidth}{%
    \centering%
    #1%
  }\par
  \parbox[c][6cm][c]{\textwidth}{%
    \centering%
    #3%
  }\par
  \clearpage
}%

\newsavebox{\scaleandresizetmp}
\newcommand{\scaleandresize}[1]{%
  \sbox{\scaleandresizetmp}{#1}%
  \ifdim 16\wd\scaleandresizetmp < \dimexpr \textwidth - 2cm \relax
    \scalebox{16}{#1}%
  \else
    \resizebox{\dimexpr \textwidth - 2cm \relax}{\dimexpr 16\ht\scaleandresizetmp}{#1}%
  \fi
}

% \pil{\command}{navn}{tekst}
% \command{\retning}{n}
\def\pil#1#2#3{%
  \newcommand{#1}[2]{%
    \Repeat{##2}{%
      \skilt{%
        \includegraphics[angle=##1,origin=c]{pil.pdf}%
      }{\scaleandresize{#2}}{#3}%
    }%
  }%
}%

\newcommand{\op}{90}
\newcommand{\hojre}{0}
\newcommand{\venstre}{180}
\newcommand{\ned}{270}
